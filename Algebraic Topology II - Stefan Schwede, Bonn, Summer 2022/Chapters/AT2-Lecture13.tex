% Lecture 13

\lecture[]{2022-05-30}

\noindent
The next theorem looks a lot like the dualization of the previous one (theorem \ref{proposition:homotopy-category-has-coproducts}), but as the axioms for cofibration categories are not self dual (contrary to the ones for model categories) we have to prove everything again (although the proof is surprisingly similar).

\begin{theorem}
Let $\Cc$ be a cofibration category.
\begin{rmnumerate}
    \item The localization functor $\gamma:\Cc\to\Ho(\Cc)$ preserves terminal objects.
    \item Let $I$ be any set. Suppose that $\Cc$ has $I$-indexed products and weak equivalences are closed under $I$-indexed products. Then $\gamma:\Cc\to\Ho(\Cc)$ preserves $I$-indexed products. In particular $\Ho(\Cc)$ has $I$-indexed products.
\end{rmnumerate}
\end{theorem}

\begin{proof}
(i) Let $*\in\Cc$ be a terminal object.\rightnote{We\vspace{-0.3ex} could of course just prove (ii).} Call $p_A:A\to*$ the unique $\Cc$-morphism from an object $A$.
Any morphism from $A$ to $*$ in $\Ho(\Cc)$ is of the form $\gamma(\tau)^\inv\circ\gamma(f)$ for some morphism $f:A\to Z$ and $\tau:*\xto{\sim}Z$ in $\Cc$ with $\tau$ a weak equivalence.
The unique $\Cc$-morphism $p_Z:Z\to*$ is left inverse to $\tau$, so $p_Z$ is a weak equivalence by 2-out-of-3 and $\gamma(p_Z)=\gamma(\tau)^\inv$. Then
\[\gamma(\tau)^\inv\circ\gamma(f)=\gamma(p_Z)\circ\gamma(f)=\gamma(p_Zf)=\gamma(p_A),\]
hence there is exactly one morphism from $A$ to $*$ in $\Ho(\Cc)$.

(ii) Let $\cb{X_i}_{i\in I}$ be a $I$-family of $\Cc$-products. Let $\prod_{i\in I}X_i$ be a product in $\Cc$, with $p_j:\prod_{i\in I}X_i\to X_i$ the canonical morphisms. We want to show that for all $A\in\ob(\Cc)$, the map
\begin{align*}
    \Ho(\Cc)(A,\ts\prod_{i\in I}X_i)&\to\ts\prod_{j\in I}\Ho(\Cc)(A,Xj)\\
    \psi&\mapsto(\gamma(p_j)\circ\psi)_{j\in I}
\end{align*}
is bijective.

Surjectivity. We consider an $I$-family of $\Ho(\Cc)$-morphisms $\psi_j:A\to X_j$ for $j\in I$. The calculus of fractions provides $\Cc$-morphisms $f_j:A\to Z_j$ and acyclic cofibrations $s_j:X_j\acco Z_j$ such that
\[\psi_j=\gamma(s_j)^\inv\circ\gamma(f_j).\]
Then
\[\ts\prod s_j:\ts\prod_{j\in I}X_j\xto{\sim}\ts\prod_{j\in I}Z_j\]
is another weak equivalence by assumption.

We claim that the morphism $\gamma(\prod s_j)^\inv\circ\gamma((f_i)_{i\in I}):A\to\prod_{i\in I}X_i$ is a preimage of the family $(\psi_j)_{j\in I}$. In other words, the following is a commutative diagram in $\Cc$
\[
\begin{tikzcd}[column sep=large]
A \arrow[r, "{(f_i)_{i \in I}}"] \arrow[dr, "f_j", swap] & \prod_{i \in I} Z_i \arrow[d, "p_j"] \arrow[rr, "\Pi_j s_j", "\sim"'] & & \prod_{i \in I} X_i \arrow[d, "p_j"]\\
 & Z_j & & X_j \arrow[ll, "s_j", "\sim"']
\end{tikzcd}
\]
Indeed,
\[\gamma(p_j)\circ\gamma(\ts\prod s_j)^\inv\circ\gamma((f_i)_{i\in I})=\gamma(s_j)^\inv\circ\gamma(p_j)\circ\gamma((f_i)_{i\in I})=\gamma(s_j)^\inv\circ\gamma(f_j)=\psi_j.\]

Injectivity. While surjectivity is (surprisingly) similar to the proof of surjectivity in theorem \ref{proposition:homotopy-category-has-coproducts}, injectivity is more difficult...

\medskip
\todo[inline,color=red]{TODO}
\smallskip

\end{proof}

\section{Additive Categories}

In this section we derive an important property of the derived category of an orthogonal ring spectrum: it is an additive category.\todo{Maybe expand}

\begin{definition}
A category $\Cc$ is \tit{preadditive} if it has a zero object, finite coproducts and for all pair of objects $X,Y$ the morphisms
\[p_1=\id+0:X\ts\coprod Y\to X,\ p_2=0+\id:X\ts\coprod Y\to Y\]
make $X\coprod Y$ into a product of $X$ and $Y$, i.e. for all $\Cc$-objects $A$, the map
\begin{align*}
    \Cc(A,X\ts\coprod Y)&\to\Cc(A,X)\times\Cc(A,Y)\\
    f&\mapsto(p_1f,p_2f)
\end{align*}
is bijective.
\end{definition}

\begin{examples}
Examples are of course abelian groups, modules over a ring, abelian monoids, but also $D(R)$ for $R$ a ring spectrum.
\end{examples}

\begin{remark}
Our definition of preadditive categories is a bit different than the usual definition, i.e. an $\Ab$-enriched category with a zero object. In fact it is strictly related, as the next proposition shows (although not the same, and indeed our preadditive categories are usually called \tit{semiadditive categories} instead). It could be argued that our definition is the \enquote{right} one to give, as it makes clear that being an additive category is a property and does not come from having additional structure.
\end{remark}

\begin{construction}
Let $\Cc$ be preadditive. Let $a,b:A\to X$ be two $\Cc$-morphisms. We denote by $a\perp b$ the unique morphism $A\to X\coprod X$ such that $p_1\circ(a\perp b)=a$ and $p_2\circ(a\perp b)=b$. The \tit{sum} of $a$ and $b$ is defined by
\[a+b=\nabla(a\perp b)=(\id+\id)(a\perp b).\]
\end{construction}

\begin{proposition}\label{proposition:semiadditive-categories}
Let $\Cc$ be a preadditive category.
\begin{rmnumerate}
    \item For all $\Cc$-objects $A,X$ the binary operation $+$ makes $\Cc(A,X)$ into an abelian monoid. Composition is biadditive.
    \item If the shear map $\nabla\perp p_1:X\coprod X\to X\coprod X$ is an isomorphism, then $\Cc(A,X)$ is an abelian groups.
    \item Let $F:\Cc\to\operatorname{AbMon}$ be a functor that preserves zero objects and finite coproducts. Then for all $\Cc$-objects $A,X$ and all $a\in F(a)$, the map
    \[\Cc(A,X)\to F(X),\ f\mapsto f_*(a)=F(f)(a)\]
    is a monoid morphism
\end{rmnumerate}
\end{proposition}

\begin{proof}
Eh.

\medskip
\todo[inline,color=red]{TODO}
\smallskip

\end{proof}

\begin{definition}
An \tit{additive} category $\Cc$ is a preadditive category such that for all objects $X$ the morphism $\nabla\perp p_1:X\prod X\to X\prod X$ is an isomorphism (in particular, it is $\Ab$-enriched by proposition \ref{proposition:semiadditive-categories}).
\end{definition}

\begin{theorem}
Let $R$ be a ring spectrum.
\begin{rmnumerate}
    \item The\rightnote{\upshape Actually, arbitrary products are also preserved, but we don't show that.} localization functor $\gamma:\RMod\to D(R)$ preserves arbitrary coproducts and finite products. In particular, $D(R)$ has arbitrary coproducts and finite products.
    \item The derived category $D(R)$ is additive.
\end{rmnumerate}
\end{theorem}

\begin{proof}
Eh$\times2$.

\medskip
\todo[inline,color=red]{TODO}
\smallskip

\end{proof}
