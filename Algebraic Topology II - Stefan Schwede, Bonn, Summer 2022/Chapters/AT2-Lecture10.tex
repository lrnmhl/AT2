% Lecture 10

\section{Cofibration Categories}

\lecture[]{2022-05-16}

\begin{definition}
A \tit{cofibration category}\rightnote{This notion is clearly modeled after the one of model category, and was introduced (as fibration cate-\\gories, by Ken Brown) when model categories had already an established theory; we consider cofib-\\ration categories because they are simpler, but the notion is \href{https://mathoverflow.net/questions/62386/examples-of-brown-cofibration-categories-that-are-not-quillen-model-categories}{of interest in its own right}.} is a triple $(\Cc,\Cof,W)$ consisting of a category $\Cc$ and two distinguished classes of morphisms the cofibrations $\Cof$ and the weak equivalences $W$, satisfying the following axioms.
\begin{itemize}
    \item[(C1)] Isomorphisms are cofibrations and weak equivalences. Cofibrations are closed under composition. There is an initial object $\emptyset$ and all initial morphisms are cofibrations.
    \item[(C2)] If $g$ and $f$ are composable and two out of $g$, $f$ and $gf$ are weak equivalences, then so is the third.
    \item[(C3)] Given a cofibration $i:A\to B$ and any morphism $f:A\to C$, there is a pushout in $\Cc$
    \[
    \begin{tikzcd}
    A \ar[d,"i"] \ar[r,"f"] & C \ar[d,"j"]\\
    B \ar[r] & D
    \end{tikzcd}
    \]
    and $j$ is again a cofibration. If in addition, $i$ is a weak equivalence, then so is $j$.
    \item[(C4)] Every morphism in $\Cc$ can be factored as a composite of a cofibration followed by a weak equivalence.
\end{itemize}

An \tit{acyclic cofibration} (or \tit{trivial} cofibration) is a morphism that is both a cofibration and a weak equivalence. Cofibration are denoted by $a\tto b$, weak equivalences by $a\xto{\sim}b$ and acyclic cofibrations by $a\acco b$.
\end{definition}

An immediate consequence of the definitions (in particular, of C1 and C3) is that finite coproducts exist in $\Cc$. Moreover, the two canonical morphisms
\[A\to A\amalg B\leftarrow B\]
are cofibrations. We can see that the coproduct of two cofibrations $i:A\to B$ and $i':A'\to B'$ is a cofibration, by applying C3 twice and then C2:
\[
    \begin{tikzcd}
    A \ar[d] \ar[r,"i"] & B \ar[d]\\
    A\amalg A' \ar[r,"i\amalg A'"] & B\amalg A'
    \end{tikzcd}\quad
    \begin{tikzcd}
    A' \ar[d] \ar[r,"i'"] & B' \ar[d]\\
    B\amalg A' \ar[r,"B\amalg i'"] & B\amalg B'
    \end{tikzcd}
\]
\[
\begin{tikzcd}
A\amalg A' \ar[dr,"i\amalg A'"'] \ar[rr,"i\amalg i'"] & & B\amalg B'\\
 & B\amalg A' \ar[ur,"B\amalg i'"']
\end{tikzcd}
\]
Similarly, the coproduct of two acyclic cofibrations is an acyclic cofibration.

The \tit{homotopy category} of a cofibration is the localization\rightnote{Note that this depends only on $W$ and not on $\Cof$.}
\[\gamma:\Cc\to\Ho(\Cc)\]
at the class $W$.

\begin{definition}
A morphism of orthogonal spectra $f:A\to B$ is a h-cofibration (of spectra) if it has the homotopy extension property: for every morphism of orthogonal spectra $\phi:B\to X$ and every homotopy $H:A\smsh[0,1]_+\to X$ starting with $\phi\circ f:A\to X$, there is a homotopy $\bar H:B\smsh[0,1]_+\to X$ that extends $H$ and begins with $\phi$, i.e. such that
\[\bar H\circ(f\smsh[0,1]_+)=H\text{ and }\bar H\circ(-\smsh0)=\phi.\]
In particular, h-cofibrations of orthogonal spectra are levelwise h-cofibrations of spaces.
\end{definition}

There is a universal test case, i.e. when $X=M_f=(A\smsh[0,1]_+)\cup_f B$, the mapping cone of $f$. In this case we have
\[
\begin{tikzcd}
A \arrow[r, "f"] \arrow[d, "- \smsh 0", swap] & B \arrow[d] \arrow[rrddd, "\phi", bend left]\\
{A \smsh [0,1]_+} \arrow[rrrdd, bend right=20] \arrow[r] & {(A \smsh [0,1]_+) \cup_f B = M_f} \arrow[dr]\\
 & &[-4em] {B \smsh [0,1]_+} \arrow[dr, dashed, "\bar H"]\\
 & & &[+1em] X
\end{tikzcd}
\]
so $f$ is a h-cofibration if and only if the canonical morphism
\[A\smsh[0,1]_+\cup_f B\to B\smsh[0,1]_+\]
has a retraction.

There is an adjoint form of the HEP. Any given homotopy extension data $(\phi,H)$ adjoins to a commutative square
\[
\begin{tikzcd}[column sep=large]
A \arrow[r, "\tilde{H}"] \arrow[d, "f", swap] & {X^{[0,1]}} \arrow[d, "\ev_0"]\\
B \arrow[ur, "\tilde{H}'", dashed] \arrow[r, "\phi", swap] & X
\end{tikzcd}
\]
The h-cofibrations are precisely the class of morphism that have the left lifting property against all evaluations $\ev_0:X^{[0,1]}\to X$ for all $X\in\Sp$.

\begin{exercise}[AT2Sheet6.1]
There is an important general nonsense fact: let $\Ee$ be a class of morphism in a category $\Cc$. Let $^\perp\Ee$ be the class of morphism with LLP against all morphisms in $\Ee.$
Then $^\perp\Ee$ is closed under
\begin{itemize}
    \item cobase change,
    \item sequential composition,
    \item composition,
    \item retract.
\end{itemize}
In particular, the h-cofibrations of orthogonal spectra have all these closure properties.
\end{exercise}

We can now put a cofibration structure on modules over a ring spectra.

\begin{theorem}\label{theorem:cofibration-structure-on-modules-over-a-ring-spectra}
The following data defines a cofibration structure on the category of left modules over an orthogonal ring spectrum:
\begin{itemize}
    \item weak equivalences are stable equivalences of underlying orthogonal spectra,
    \item cofibrations are h-cofibrations of underlying orthogonal spectra.
\end{itemize}
\end{theorem}

\begin{proof}
We just have to go diligently through the axioms.\rightnote{See AT2Sheet6.2 (and 6.3) for other (sometimes) inter-\\esting examples\\ of (co)fibration categories.}

(C1) All isomorphisms are clearly cofibrations and weak equivalences. There is an initial object and it is easy to see that the morphism $*\to X$ is an h-cofibration. Finally, h-cofibrations are stable under composition by AT2Sheet6.1.

(C2) Clearly holds.

(C3) Cofibrations are stable under cobase change again by AT2Sheet6.1 and acyclic cofibrations are stable under cobase change by proposition \ref{proposition:preservation-of-stable-equivalences}.

(C4)\rightnote{The factorization property is usually much more difficult to prove for model structures and generally relies on the \href{https://ncatlab.org/nlab/show/small+object+argument}{small object argument}.} Let $f:X\to Y$ be a morphism of $R$-modules. We factor it through the mapping cylinder as the composite of the mapping cylinder inclusion $-\smsh0:x\to (X\smsh[0,1]_+)\cup_f Y$ and the projection $(X\smsh[0,1]_+)\cup_f Y\to Y$. This projection is a homotopy equivalence of orthogonal spectra, hence a stable equivalence. We consider now the pushout square
\[
\begin{tikzcd}
X\coprod X \ar[d,tail,"(-\smsh0)\amalg(-\smsh1)"'] \ar[r,tail,"\id_X\amalg f"] & X\coprod Y \ar[d,tail]\\
X\smsh[0,1]_+ \ar[r,""] & (X\smsh[0,1]_+)\cup_f Y
\end{tikzcd}
\]
The left vertical morphism is an h-cofibration, hence so is the right vertical morphism. Since the inclusion $X\mapsto X\coprod Y$ is a cofibration, we can conclude that the mapping cylinder inclusion also is $-\smsh0:x\to (X\smsh[0,1]_+)\cup_f Y$.
\end{proof}

\begin{remark}
The stable equivalences of $R$-modules can be complemented by various different classes of cofibrations into cofibration categories, e.g.
\begin{itemize}
    \item HEP internal to $R$-modules,
    \item cofibrations in some model category structure.
\end{itemize}
Overall, it is the opinion of the professor that the approach we are taking is the shortest path.
\end{remark}

\begin{proposition}[Gluing lemma]
Let $\Cc$ be a cofibration category and
\[
\begin{tikzcd}
A \ar[d,"\sim"] & B \ar[l,tail,"i"'] \ar[d,"\sim"] \ar[r,"f"] & C \ar[d,"\sim"]\\
A' & B' \ar[l,tail,"i'"] \ar[r,"f'"'] & C
\end{tikzcd}
\]
such that $i$ and $i'$ are cofibrations and all the vertical morphisms are weak equivalences. Then the induced morphism
\[A\cup_B C\to A'\cup_{B'}C'\]
is a weak equivalence.

As a special case the diagram
\[
\begin{tikzcd}
A \ar[d,eq] & B \ar[l,tail,"i"] \ar[d,eq] \ar[r,eq] & B \ar[d,"\sim","f"']\\
A & B \ar[l,tail,"i"] \ar[r,"f","\sim"'] & C
\end{tikzcd}
\]
yields
\[
\begin{tikzcd}
B \ar[d,"\sim","f"'] \ar[r,tail] & A \ar[d,"\sim"]\\
C \ar[r,tail] & A\cup_B C
\end{tikzcd}
\]
\end{proposition}

\begin{proof}[(Not quite a) Proof]
This is an important technical lemma, related to \href{https://ncatlab.org/nlab/show/homotopy+pullback}{homotopy pushouts}, but we do not present a proof here (it is a rather dry proof, not very illuminating), a reference is \cite[Lemma 1.4.3]{rad}. Note that the special case implies that our cofibration categories are always \href{https://ncatlab.org/nlab/show/proper+model+category#left_proper_model_categories}{left proper}.
\end{proof}

\subsection{The homotopy relation}

\begin{definition}
Let $\Cc$ be a cofibration category. A \tit{cylinder object} for an object $A$ is a quadruple $(I,i_0,i_1,p)$\rightnote{The cylinder object is far from unique!} where $I$ is a $\Cc$-object, the diagram
\[
\begin{tikzcd}[row sep=small]
A \arrow[dr, "i_0", swap] \arrow[rrd, bend left, "\id"]\\
 & I \arrow[r, "p"] & A\\
A \arrow[ur, "i_1"] \arrow[rru, bend right, "\id"']
\end{tikzcd}
\]
commutes, $p$ is a weak equivalence and $i_0+i_1:A\amalg A\to I$ is a cofibration (note that $i_0$ and $i_1$ are weak equivalences by 2-out-of-3).
\end{definition}

Cylinder objects always exist: it suffices to use (C4) to factor the fold map $\nabla=\id+\id:A\amalg A\to A$ into
\[
\begin{tikzcd}
A\amalg A \ar[r,tail,"i_0+i_1"] &[+1em] I \ar[r,"\sim"] & A.
\end{tikzcd}
\]

\begin{definition}
Two morphisms $f,g:A\to Z$ are homotopic\rightnote{This definition of homotopy has a degree of flexibility because cylinder objects are not unique.} if there is a cylinder object $(I,i_0,i_1,p)$ for $A$ and a morphism (the homotopy):
\[H:I\to Z\]
such that $H\circ i_0=f$ and $H\circ i_1=g$. In this case we write $f\simeq g$.
\end{definition}

By the usual argument, the localization functor $\gamma:\Cc\to\Ho(\Cc)$ takes the same value on homotopic morphisms.

\begin{example}
Considering the cofibration structure on $\RMod$ given in theorem \ref{theorem:cofibration-structure-on-modules-over-a-ring-spectra}, the following is a cylinder object
\[M\vee M=M\smsh\cb{0,1}_+\rightarrowtail M\smsh[0,1]_+\xto{p}M\]
where the first map is a h-cofibration and $p$ is the map collapsing $[0,1)$ to $*$, which is a stable equivalence. So the morphisms in $\RMod$ that are homotopic in the classical/concrete sense (via $M\smsh[0,1]_+$) are also homotopic in the abstract sense. But note that the converse is not true! Abstract homotopy is strictly more general than concrete homotopy.
\end{example}

\begin{proposition}
Let $A$ and $Z$ be objects in a cofibration category $\Cc$.
\begin{rmnumerate}
\item \enquote{Being homotopic} is an equivalence relation on $\Cc(A,Z)$.
\item Postcomposition with any morphism $\phi:Z\to Z'$ preserves the homotopy relation.
\item Let $f,g:A\to Z$ be homotopic and let $\phi:\bar A\to A$ be any morphism. Then there is an acyclic cofibration $s:Z\acco Z'$ such that $sf\phi\simeq sg\phi$ are homotopic.
\item Let $f,g:A\to Z$ and $\tau:\bar A\to A$ a weak equivalence. If $f\tau\simeq g\tau:\bar A\to Z$, then $f\sim g$.
\end{rmnumerate}
\end{proposition}

\begin{proof}\renewcommand{\qed}{\hfill\tit{To be continued...}}
(i) Reflexivity. Let $(I,i_0,i_1,p)$ be a cylinder object for $A$. Then set
\[H=f\circ p.\]
We have 
\[Hi_0=fpi_0=f=fpi_1=Hi_1,\]
so $H$ is a homotopy from $f$ to $f$.

Symmetry. Let $H:I\to Z$ be a homotopy from $f$ to $g$ based on $(I,i_0,i_1,p)$. Then the same $H:I\to Z$ is a homotopy from $g$ to $f$ based on $(I,i_1,i_0,p)$.

Transitivity. This one is annoying.

\medskip
\todo[inline,color=red]{finish typing up!}
\smallskip

\end{proof}
